\documentclass{article}

\usepackage{arxiv}

\usepackage[utf8]{inputenc} % allow utf-8 input
\usepackage[T1]{fontenc}    % use 8-bit T1 fonts
\usepackage[hidelinks]{hyperref}       % hyperlinks
\usepackage{url}            % simple URL typesetting
\usepackage{booktabs}       % professional-quality tables
\usepackage{amsmath,amssymb,amsthm}
\usepackage{amsfonts}       % blackboard math symbols
\usepackage{nicefrac}       % compact symbols for 1/2, etc.
\usepackage{makecell}
\usepackage{microtype}      % microtypography
\usepackage{mathrsfs}
\usepackage{graphicx}
\usepackage{doi}
\usepackage{float}
\usepackage{acronym}
\usepackage{listings}
\usepackage{multicol}
\usepackage{tikz}
\usepackage[dvipsnames]{xcolor}

\usetikzlibrary{trees}

\def\checkmark{\tikz\fill[scale=0.4](0,.35) -- (.25,0) -- (1,.7) -- (.25,.15) -- cycle;}
\def\cross{\tikz\draw[scale=0.3, black, line width=0.3mm](0,0) -- (1,1) -- (0.5,0.5) -- (0,1) -- (1,0) -- (0.5,0.5);}

\newacro{abm}[ABM]{Agent-Based Model}
\newacro{cabm}[CABM]{Cellular Agent-Based Model}
\newacro{ca}[CA]{Cellular Automaton}
\newacro{ib}[IB]{Individual-Based}
\newacro{ode}[ODE]{Ordinary Differential Equation}
\newacro{pde}[PDE]{Partial Differential Equation}
\newacro{dnn}[DNN]{Deep Neural Network}
\newacro{mbs}[MBS]{Many-Body Simulation}
\newacroplural{ca}[CA]{Cellular Automata}

\newcommand{\todo}[1]{\colorbox{WildStrawberry}{\textcolor{white}{#1}}}

\newcommand{\R}{\mathbb{R}}

\title{
    Cellular Agent-Based Modelling Frameworks are not flexible enough
    % The Significance of Flexibility in the\\
    % Design of Agent-Based Models
}

%\date{September 9, 1985}	% Here you can change the date presented in the paper title
%\date{} 					% Or removing it

\author{
    \href{https://orcid.org/0009-0001-0613-7978}{
        \includegraphics[scale=0.06]{orcid.pdf}
        \hspace{1mm}Jonas Pleyer
    }
    \thanks{
        \href{https://jonas.pleyer.org}{jonas.pleyer.org},
        \href{https://cellular-raza.com}{cellular-raza.com}
    }\\
	Freiburg Center for Data-Analysis and Modeling\\
	University of Freiburg\\
	\texttt{jonas.pleyer@fdm.uni-freiburg.de} \\
	%% examples of more authors
	\And
	\href{https://orcid.org/0000-0002-6371-4495}{
        \includegraphics[scale=0.06]{orcid.pdf}
        \hspace{1mm}Christian Fleck
    }\\
	Freiburg Center for Data-Analysis and Modeling\\
	University of Freiburg
}

% Uncomment to remove the date
%\date{}

% Uncomment to override  the `A preprint' in the header
\renewcommand{\headeright}{An Opinion Piece}
\renewcommand{\undertitle}{An Opinion Piece}
%\renewcommand{\undertitle}{Technical Report}
\renewcommand{\shorttitle}{The Significance of Flexibility in the Design of Agent-Based Models}

\usepackage{enumitem}
\setlist{nolistsep}

%%% Add PDF metadata to help others organize their library
%%% Once the PDF is generated, you can check the metadata with
%%% $ pdfinfo template.pdf
\hypersetup{
pdftitle={The Significance of Flexibility in the Design of Agent-Based Models},
pdfsubject={q-bio.NC, q-bio.QM},
pdfauthor={Jonas Pleyer, Christian Fleck},
pdfkeywords={},
}

% Change numbering of equations
% \numberwithin{equation}{section}

% MAKE TITLES IN THEOREMS BOLD
\makeatletter
\def\th@plain{%
  \thm@notefont{}% same as heading font
  \itshape % body font
}
\def\th@definition{%
  \thm@notefont{}% same as heading font
  \normalfont % body font
}
\makeatother

\begin{document}

\maketitle

\paragraph{Title Building Blocks}
\begin{enumerate}
    \item \textbf{\acl{abm}}
    \item \textbf{Frameworks} (possibly optional, "models" alone only refers to single-purpose studies;
        frameworks refers to the libraries that aim to provide something abstract enough such that
        it can be reused)
    \item \textbf{Flexible/Flexibility} ("not sufficient/enough")
    \item \textbf{Cellular}
        \textit{We restrict ourselves cellular systems only. Covering all of Biology is too much.}
\end{enumerate}

%###################################################################################################
\begin{abstract}
\end{abstract}

% keywords can be removed
\keywords{Individual-Based \and Cell \and Biology \and Dynamical System}

\multicols{2}

%###################################################################################################
\section{Introduction}

\paragraph{Usage of \acl{abm}}
\begin{itemize}
    \item \cite{Pleyer2023} \acl{abm} are widely used for many problems
    \item Common choices for cellular representation (i.e. spherical cells, off-lattice approach)
    \item More biology than models available
\end{itemize}

\textbf{I argue that \acp{abm} need to be more flexible to support more use-cases and become a tool
that researchers can experiment with more freely.}

\section{Principles of Foundational Tools}

\paragraph{Introduce some really widespread techniques}
\begin{enumerate}
    \item \acfp{ode}
    \item \acfp{pde}
    \item \acfp{mbs}
    \item \acfp{dnn}
\end{enumerate}

\paragraph{Observations about these tools}
\begin{itemize}
    \item Very generalized concepts; specific usage/model brings forth plethora of results
    \item They have one two things: A mathematical description that everyone agrees on; A solving
        scheme that everyone agrees on (with variations of course)
    \item $\Rightarrow$ identify overarching principles:
    \item Strong results
    \item Good usability (ie. tools \& Performance)
    \item \textbf{Generalized formulation} $\longrightarrow$ specific use-case;\\
        However this is not how they got developed (developed from collection of examples)
\end{itemize}

\paragraph{General Modeling Approach}
\begin{itemize}
    \item Adapt model to biological reality
    \item Ability to reduce the model until nothing is left
    \item No inheritance of parameters/magic numbers
\end{itemize}

\begin{figure}[H]
    \centering
    \includegraphics[width=0.9\columnwidth]{example-image-a}
    \caption{TODO (Schema: Generalized Modeling approach)}
\end{figure}

\section{Cellular \aclp{abm}}

\paragraph{Constructing a computational Model - 2 Options}
\begin{enumerate}
    \item Use existing \ac{abm} (Framework)
    \item Build something from scratch (Possibly even with some really general \ac{abm}
        framework such as netlogo)
\end{enumerate}
% They often combine simulation techniques such as \acp{ode} and \acp{pde}

\paragraph{Problems of Approach 1}
\begin{itemize}
    \item Inheritance of existing parameters
    \item Does the existing model match the use-case? Or are researchers trying to make it conform?
    \item Once work has started and \ac{abm} is chose, often not simply able to switch
    \item Overall not suited for Occam's Razor (most simple approach)
\end{itemize}

\paragraph{Problems of Approach 2}
\begin{itemize}
    \item Can be good choice if model can be easily constructed and \ac{abm} framework allows for
        desired representation
    \item Lots of (human) effort to construct computational model
    \item Often limited reusability
\end{itemize}

\section{Conclusion}
\begin{itemize}
    \item Democratization is good but only relevant if flexibility and performance is given \cite{Johnson2025}
    \item \cite{Pleyer2025} \texttt{cellular\_raza} solves some of these problems
\end{itemize}

\onecolumn
\bibliographystyle{IEEEtran}
\bibliography{references}

\end{document}
