\documentclass{article}

\usepackage{arxiv}

\usepackage[utf8]{inputenc} % allow utf-8 input
\usepackage[T1]{fontenc}    % use 8-bit T1 fonts
\usepackage[hidelinks]{hyperref}       % hyperlinks
\usepackage{url}            % simple URL typesetting
\usepackage{booktabs}       % professional-quality tables
\usepackage{amsmath,amssymb,amsthm}
\usepackage{amsfonts}       % blackboard math symbols
\usepackage{nicefrac}       % compact symbols for 1/2, etc.
\usepackage{makecell}
\usepackage{microtype}      % microtypography
\usepackage{mathrsfs}
\usepackage{multicol}
\usepackage{graphicx}
\usepackage{doi}
\usepackage{float}
\usepackage{acronym}
\usepackage{listings}
\usepackage[obeyclassoptions]{standalone}
\usepackage{tikz}
\usepackage[dvipsnames]{xcolor}

\newacro{abm}[ABM]{Agent-Based Model}
\newacro{ca}[CA]{Cellular Automaton}
\newacro{ib}[IB]{Individual-Based}
\newacro{ode}[ODE]{Ordinary Differential Equation}
\newacro{pde}[PDE]{Partial Differential Equation}
\newacro{dnn}[DNN]{Deep Neural Network}
\newacro{mbs}[MBS]{Many-Body Simulation}
\newacroplural{ca}[CA]{Cellular Automata}
\newacro{gpu}[GPU]{Graphical Processing Unit}

\newcommand{\todo}[1]{\colorbox{WildStrawberry}{\textcolor{white}{#1}}}

\definecolor{COLOR1}{HTML}{6bd2db}
\definecolor{COLOR2}{HTML}{0ea7b5}
\definecolor{COLOR3}{HTML}{0c457d}
\definecolor{COLOR4}{HTML}{ffbe4f}
\definecolor{COLOR5}{HTML}{e8702a}

\newcommand{\R}{\mathbb{R}}

\title{
    Agent-Based Modelling in Cellular Biology\\Are we flexible yet?
    % The Significance of Flexibility in the\\
    % Design of Agent-Based Models
}

%\date{September 9, 1985}	% Here you can change the date presented in the paper title
%\date{} 					% Or removing it

\author{
    \href{https://orcid.org/0009-0001-0613-7978}{
        \includegraphics[scale=0.06]{orcid.pdf}
        \hspace{1mm}Jonas Pleyer
    }
    \thanks{
        \href{https://jonas.pleyer.org}{jonas.pleyer.org},
        \href{https://cellular-raza.com}{cellular-raza.com}
    }\\
	Freiburg Center for Data-Analysis and Modeling\\
	University of Freiburg\\
	\texttt{jonas.pleyer@fdm.uni-freiburg.de} \\
	%% examples of more authors
	\And
	\href{https://orcid.org/0000-0002-6371-4495}{
        \includegraphics[scale=0.06]{orcid.pdf}
        \hspace{1mm}Christian Fleck
    }\\
	Freiburg Center for Data-Analysis and Modeling\\
	University of Freiburg
}

% Uncomment to remove the date
%\date{}

% Uncomment to override  the `A preprint' in the header
\renewcommand{\headeright}{An Opinion Piece}
\renewcommand{\undertitle}{An Opinion Piece}
%\renewcommand{\undertitle}{Technical Report}
\renewcommand{\shorttitle}{Agent-Based Modelling in Cellular Biology - Are we flexible yet?}

\usepackage{enumitem}
\setlist{nolistsep}

%%% Add PDF metadata to help others organize their library
%%% Once the PDF is generated, you can check the metadata with
%%% $ pdfinfo template.pdf
\hypersetup{
pdftitle={Agent-Based Modelling in Cellular Biology - Are we flexible yet?},
pdfsubject={q-bio.NC, q-bio.QM},
pdfauthor={Jonas Pleyer, Christian Fleck},
pdfkeywords={},
}

% Change numbering of equations
% \numberwithin{equation}{section}

% MAKE TITLES IN THEOREMS BOLD
\makeatletter
\def\th@plain{%
  \thm@notefont{}% same as heading font
  \itshape % body font
}
\def\th@definition{%
  \thm@notefont{}% same as heading font
  \normalfont % body font
}
\makeatother

\begin{document}

\maketitle

% \paragraph{Title Building Blocks}
% \begin{enumerate}
%     \item \textbf{\acl{abm}}
%     \item \textbf{Frameworks} (possibly optional, "models" alone only refers to single-purpose studies;
%         frameworks refers to the libraries that aim to provide something abstract enough such that
%         it can be reused)
%     \item \textbf{Flexible/Flexibility} ("not sufficient/enough")
%     \item \textbf{Cellular}
%         \textit{We restrict ourselves cellular systems only. Covering all of Biology is too much.}
% \end{enumerate}

%###################################################################################################
\begin{abstract}
    Cellular \aclp{abm} are widely used tools to describe biological systems.
    Over the course of the past years, many modeling tools have emerged which solve particular
    research questions.
    In this short opinion piece, we argue that existing frameworks lack flexibility compared to the
    inherent underlying complexity that they should be able to represent.
    By extracting overarching principles of widely used software solutions across multiple domains we
    compare these with currently existing \acp{abm}.
    I come to the conclusion that existing \acp{abm} lack in flexibility which hinders overall
    progress of the field.
\end{abstract}

% keywords can be removed
\keywords{Individual-Based \and Cell \and Biology \and Dynamical System}

\begin{multicols}{2}

%###################################################################################################
\section{Introduction}

\paragraph{Usage of \acl{abm}}
\acp{abm} have become indispensable tools in the study of complex systems.
Their applications cover topics in Ecology~\cite{Grimm2013}, Social Sciences~\cite{Bankes2002},
Autonomous Cars~\cite{Karolemeas2024}, Spread of COVID infections~\cite{Shattock2022} and many more.
They utilize descriptions on the level of individual entities such as cells, organisms or humans
- often called agents - to build up models of complex systems.
As such they aim to study the emergence of collective behaviors as a result of \ac{ib} interactions.
In the field of biology, cells can be considered the fundamental building blocks of nature,
comprising many complex systems such as bacterial communities~\cite{Nagarajan2022},
organs~\cite{DuttaMoscato2014}, plants~\cite{Merks2011} and tissue patterning~\cite{Thorne2007}.
They are particularly suited to capture effects such as heterogeneity, spatial effects and provide
a natural way to combine cellular building blocks.

In order to quickly construct new simulations, many modeling frameworks have emerged which simplify
the process of model design~\cite{Pleyer2023}.
However, despite their wide applicability, most of these models rely on a fixed cellular
representation.
Furthermore, these models often come with a large set of parameters that need to be specified in
order to obtain a working simulation which results in problems with respect to interpretability and
parameter estimation of the model.

With this text, we argue that cellular \acp{abm} need to become more flexible with respect to design
of agents and environment in order to tackle the aforementioned challenges.
We further argue that many groundbreaking modeling techniques have been preceded by strong
results and made popular by generalist tools which are able to solve a whole class of problems,
thereby enabling many researchers to access the novel method.

\section{Principles of Foundational Tools}

Over the past years, a variety of foundational techniques have greatly influenced scientific
research by enabling advanced workflows and providing novel insights.
We focus on a subset of methods which are so fundamental that their usage spans across almost all
disciplines and can target a variety of problems.
Due to their foundational nature, each method comes with many tools which have been developed such
that researchers are not concerned with implementation details but can focus on the scientific
question at hand.

\paragraph{Selected Foundational Methods}
Among these are \acp{ode} which are mostly used todescribe the dynamics of various systems.
Their mathematical form is rather simple which makes it easy for computationally inclined
researchers to implement solvers.
Nevertheless, many tools exist which are frequently reused such as the Boost
library~\cite{Mulansky2011} or the Julia package
DifferentialEquations.jl~\cite{rackauckas2017differentialequations}.
When considering spatial effects, \acp{pde} provide a natural way to extend \acp{ode} although the
added structure also requries deeper mathematical knowledge and more complex numerical solvers.
Widely used tools are OpenFoam~\cite{Weller1998} for fluid simulations or
FEniCS~\cite{BarattaEtal2023,ScroggsEtal2022,BasixJoss,AlnaesEtal2014} for finite-element workflows.
\acp{mbs} excel at describing the dynamics of spatial systems involving many interacting particles.
GROMACS~\cite{Abraham2015} and LAMPPS~\cite{Thompson2022} both target classical molecular dynamics
where the latter focusses on materials modeling but also arguably provides better flexibility while
the latter is considered to be faster.
Finally, in the past years, \acp{dnn} have been applied to a variety of problems such as protein
folding~\cite{Jumper2021} or biomedical image segmentation~\cite{Ronneberger2015}.
These have been enabled by tools such as PyTorch~\cite{Ansel_PyTorch_2_Faster_2024} and
TensorFlow~\cite{tensorflow2015-whitepaper}.

% \paragraph{Introduce some really widespread techniques}
% \begin{enumerate}
%     \item \acfp{ode} Boost (C++)~\cite{Mulansky2011}, Julia~\cite{rackauckas2017differentialequations}
%     \item \acfp{pde} OpenFoam~\cite{Weller1998}, FEniCS~\cite{BarattaEtal2023,ScroggsEtal2022,BasixJoss,AlnaesEtal2014}
%     \item \acfp{mbs} LAMMPS~\cite{Thompson2022}, GROMACS~\cite{Abraham2015}
%     \item \acfp{dnn} PyTorch\cite{Ansel_PyTorch_2_Faster_2024}, TensorFlow~\cite{tensorflow2015-whitepaper}
% \end{enumerate}

\paragraph{Shared Principles}
% \textbf{Observations about these tools}
% \begin{itemize}
%     \item Very generalized concepts; specific usage/model brings forth plethora of results
%     \item They have one two things: A mathematical description that everyone agrees on; A solving
%         scheme that everyone agrees on (with variations of course)
%     \item $\Rightarrow$ identify overarching principles:
%     \item Strong results
%     \item Good usability (ie. tools \& Performance)
%     \item \textbf{Generalized formulation} $\longrightarrow$ specific use-case;\\
%         However this is not how they got developed (developed from collection of examples)
% \end{itemize}

Despite their differences in applications and technical details, all of these tools exhibit similar
unifying characteristics.
Every individual component of these tools can be formulated using mathematical methods despite the
fact that most scientific results can not be calculated by a purely mathematical approach.
This is also true for computations involving stochatic effects such as stochastic \acp{ode} or
calculations which are only approximate.
This generalized concept can be observed across all levels of complexity.
Furthermore, since many of these methods have been established for numerous years, they follow
mathematical notions which are rooted in a description that is mostly agreed upon in the
respective field, often with no or only minor modifications.
A particular problem is defined by a given a set of initially known properties (i.e. initial values,
labeled data, etc.) and a mathematical description.
In the case of \acp{ode} one could ask what the initial values of said \ac{ode} were given a set of
parameters and final values but other questions such as: "Which parameters allow me to represent
this set of datapoints most optimally?" are also valid.
The mathematical description links these quantities together and a suitable solver can be chosen
which accomplishes the goal most effectively.

The preceding observations may have only emerged once a particular method was popularized.
However, the reason that these methods became popular in the first place was due to their strong
results either in academic or industrial settings.
Tools which make it easy to exploit these methods and deliver exceptional performance are key
cornerstone for why these methods have become so popular.
In the case of \acp{dnn}, the technological advancement of training these networks on
\acp{gpu}~\cite{Raina2009} via gradient descent~\cite{Rumelhart1986} was crucial for their
breakthrough.

Finally, another important aspect is that all of these methods have the ability to be very complex
but can also be reduced until nothing is left.
This means that in the case of an \ac{ode} given by
\begin{equation}
    \partial_t \textbf{x} = \textbf{f}(t,\textbf{x})
\end{equation}
any mathematically expressable term can be inserted for $\textbf{f}(t,\textbf{x})$, allowing complex
models.
But despite this flexibility, we can still choose $\textbf{f}=0$, thus eliminating any model
properties.
These statements trivially hold true for \acp{pde} and \acp{mbs} where in the last case we can think
of either no point particles or particles with no dynamics.
It is a key feature that all of these tools allow scaling of the complexity scale without requiring
a given starting point which enables core modeling techniques such as model reduction and model
validation.

\paragraph{General Modeling Approach}
In order to construct models which can describe biological systems and connections therein, a common
workflow is usually applied.
This scheme (see Figure~\ref{fig:modeling-schematic}) involves multiple steps: A mathematical model
which get's translated into computational form, one ore more predictions of said model and finally
comparison with data to embed the model within biological reality.
Having compared a model with data, we can adjust our model in a variety of ways.
One option is to include additional effects, thus extending the model or we can simplify it, i.e.
for parameter estimation purposes.
The comparison with data is crucial in order to embed our theoretical work within biological
reality.
Furthermore, when constructing new models it is important to be able to build up models from scratch
without having to rely on existing work such that minimal models can be constructed.

\begin{figure}[H]
    \centering
    \includestandalone[width=\columnwidth]{figures/generalized-modeling}
    \caption{
        Schematic representation of a typical modeling approach.
        It is in principle possible to start at any arrow.
        A mathematical model is implemented within a computational framework which results in
        predictions that can be compared with biological data.
        This comparison nurishes our understanding of reality and allows us to formulate more
        appropriate models which starts the cycle again.
    }
    \label{fig:modeling-schematic}
\end{figure}

\section{Cellular \aclp{abm}}
% \paragraph{Constructing a computational Model - 2 Options}
% \begin{enumerate}
%     \item Use existing \ac{abm} (Framework)
%     \item Build something from scratch (Possibly even with some really general \ac{abm}
%         framework such as netlogo)
% \end{enumerate}
% They often combine simulation techniques such as \acp{ode} and \acp{pde}
When constructing an \ac{abm}, researchers are faced with the question if they should start
bottom-up, by defining core ingredients of the model by themselves or if a preexisting \ac{abm} can
be reused for their purpose.
Both approaches bear challenges and provide advantages which we want to highlight now.

\paragraph{Using an existing \acs{abm}}
% \paragraph{Problems of Approach 1}
% \begin{itemize}
%     \item Inheritance of existing parameters
%     \item Does the existing model match the use-case? Or are researchers trying to make it conform?
%     \item Once work has started and \ac{abm} is chose, often not simply able to switch
%     \item Overall not suited for Occam's Razor (most simple approach)
%     \item \cite{Ghaffarizadeh2018,Gorochowski2012,Kang2014} Common choices for cellular
%         representation (i.e. spherical cells, off-lattice approach)
% \end{itemize}
In our previous work, we have reviewed a variety of \acp{abm}~\cite{Pleyer2023}.
One common theme among these \acp{abm} is that each of them assumes a particular spatial
representation of the cell.
The most common choice is the soft-spheroid model~\cite{Ghaffarizadeh2018,Gorochowski2012} with only
a few select \acp{abm} providing support for ellipses or cylinders~\cite{Kang2014}.
Any particular choice for the spatial representation is fundamental to the dynamics which these
models are capable of describing and will carry over in other aspects such as growth, physical
interactions and intracellular processes.
Furthermore, we observed that existing \acp{abm} carry a large set of parameters which are often
necessary to specify in order to obtain the desired dynamics.
These parameters can become problematic when trying to perform a model reduction since they are
intrinsic to the chosen \ac{abm} and can not be simply removed.
By choosing a particular \ac{abm}, researchers should be concerned even more about the question if
this model fits their needs and can be fully parametrized by the given biological context.
Once work has started within a particular modeling environment, switching to a different \ac{abm} is
often tedious and costly in time.
These are significant limitations on the ability to use model reduction techniques within the
\ac{abm} and make it difficult to follow best principles such as Occam's Razor~\cite{Sober2015} for
novel model development.

\paragraph{From a clean Slate}
% \paragraph{Problems of Approach 2}
% \begin{itemize}
%     \item Can be good choice if model can be easily constructed and \ac{abm} framework allows for
%         desired representation
%     \item Lots of (human) effort to construct computational model
%     \item Often limited reusability
% \end{itemize}
The other option is to build up a model starting from a clean slate.
When deciding on this approach it is still possible to harness existing tools as long as they are
capable of describing the desired cellular properties and provide enough flexibility to perform
model reduction and parameter estimation techniques.
General-purpose tools such as Netlogo~\cite{Wilensky_NetLogo_1999} which are not specifically
tailored towards biological questions can thus be a viable option.
However, due to their general nature, they may lack desired functionalities which require additional
work to implement that falls back to the researchers.\\
The case where no existing framework is used and almost all code is written by the researchers
themselves, presents a very labor-intensive approach to solving the problem, even when utilizing
libraries for linear algebra and numerical solvers.
This approach also involves another class of challenges involving correctness of implemented
algorithms and accuracy of numerical results.\\
The resulting \ac{abm} is often highly tailored to a specific niche or the respective research
question at hand, thus limiting its ability to be reused in other contexts.

\paragraph{Building Blocks}
In contrast to the two approaches described before, there exists another path which has not been
fully realized yet.
That is to provide building blocks for cellular behaviors and let researchers be able to construct
and combine own building blocks from them.
This procedure is well known to most software developers who have authored one or more libraries.
There it is also desirable to provide reusable parts that can be combined in multiple ways.
Our own \ac{abm} cellular\_raza~\cite{Pleyer2025} aims to push further into this direction by
separating abstract mathematical concepts from numerical solvers and implemented building blocks.
However, it is too early to tell if this particular approach has the potential to become the
de-facto standard for constructing novel \acp{abm}.

\section{Conclusion}
% \begin{itemize}
%     \item Democratization is good but only relevant if flexibility and performance is given \cite{Johnson2025}
%     \item \cite{Pleyer2025} \texttt{cellular\_raza} solves some of these problems
% \end{itemize}
\acp{abm} have already proven to be useful tools in describing cellular systems and their emergent
phenomena.
Furthermore, due to their individual-based treatment of cells, they provide the most intuitive way
of bridging the gap from single-cell studies to collective phenomena.
However, in order for these tools to become widely applicable and easily reusable, a more flexible
approach which allows better reusability and customizability is required.
Frameworks which are too specialized or lack in performance will be unable to support the
exploratory nature that the diverse field of biology has to offer.
Furthermore, a more unified mathematical treatment of all approaches would allow for the
construction of more generalized libraries that can be reused by researchers across topics ranging
from microbiology over human stem cells to plant cells.
This task can be accomplished by starting with a possibly unified mathematical description or a
shared library and toolchain which allows us to derive a mathematical notion.
If the field succeeds in the constructing of at least one of them, \acp{abm} could live up to the
standards of freedom and explorability which are common in other areas of research such as \acp{dnn}
and \acp{ode}.

\end{multicols}
\bibliographystyle{IEEEtran}
\bibliography{references}

\end{document}
